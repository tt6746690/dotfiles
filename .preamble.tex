
\usepackage{marginnote}   % setting margin notes
\usepackage{caption}      % using captions
\usepackage{subcaption}   % for using sub floats within figure
\usepackage{mathtools}
\usepackage{amsmath}
\usepackage{amssymb}    % Math symbols such as \mathbb
\usepackage{amsthm}
\usepackage{pgfplots}   % plots
\usepackage{relsize}    % resizing math symbols.. i.e. \mathlarger{\Sigma}
\usepackage{bbm}        % to write indicator varialbe 
\usepackage[noline, noend]{algorithm2e} % algorithmn
% \usepackage[noline, linesnumbered, boxruled,titlenumbered, noend]{algorithm2e} % algorithmn
\SetKwProg{Fn}{Function}{}{}


\usepackage{multicol}       % writing multiple columns
\setlength{\columnsep}{0.1cm}

\usepackage{listings} % code formatting
\usepackage{color}
\usepackage{tikz}       % drawing graphics
\usetikzlibrary{automata,positioning} % for drawing automatas

\usepackage{tikz-qtree} % plotting simple trees
\usepackage{forest}     % plotting simple trees
\usepackage{enumitem}   % for alphabest enumerate
\usepackage{graphicx}   % incorporating graphics
\graphicspath{{../assets/}}

\usepackage[a4paper, total={6in, 8in}]{geometry}
\usepackage{hyperref}   % referencing
\hypersetup{colorlinks=true, linktoc=all, linkcolor=blue}

\definecolor{codegreen}{rgb}{0,0.6,0}
\definecolor{codegray}{rgb}{0.5,0.5,0.5}
\definecolor{codepurple}{rgb}{0.58,0,0.82}
\definecolor{backcolour}{rgb}{0.95,0.95,0.92}

\lstdefinestyle{mystyle}
{backgroundcolor=\color{backcolour},
commentstyle=\color{codegreen},
keywordstyle=\color{magenta},
numberstyle=\tiny\color{codegray},
stringstyle=\color{codepurple},
basicstyle=\footnotesize,
breakatwhitespace=false,
breaklines=true,
captionpos=b,
keepspaces=true,
numbers=left,
numbersep=5pt,
showspaces=false,
showstringspaces=false,
showtabs=false,
tabsize=2}
\lstset{style=mystyle}

% proper inline math display, adjust height for symbols like \sum
\everymath{\displaystyle}

% define tags for math use..
\theoremstyle{plain}% default
\newtheorem*{theorem*}{Theorem}
\newtheorem{theorem}{Theorem}[section]
\newtheorem*{corollary*}{Corollary}
\newtheorem{corollary}{Corollary}[theorem]
\newtheorem*{proposition*}{Proposition}
\newtheorem{proposition}{Proposition}
\newtheorem*{lemma*}{Lemma}
\newtheorem{lemma}{Lemma}

\newtheorem*{defn*}{Definition}
\newtheorem{defn}{Definition}[section]
\newtheorem*{example}{Example}


\theoremstyle{definition}

\theoremstyle{remark}
\newtheorem*{rem}{Remark}
\newtheorem*{note}{Note}
\newtheorem{case}{Case}

% Gives begin{solution} same formating as \begin{proof}
\newenvironment{solution}
  {\begin{proof}[Solution] \end{proof}}

\newenvironment{approach}
  {\begin{proof}[Approach] \end{proof}}

%running fraction with slash - requires math mode.
\newcommand*\rfrac[2]{{}^{#1}\!/_{#2}}

% math
\newcommand{\N}{\mathbb{N}}
\newcommand{\R}{\mathbb{R}}
\newcommand{\I}{\mathbb{I}}
% statistics
\newcommand{\pr}{\mathbb{P}}
\newcommand{\norm}{\mathcal{N}}
\newcommand{\E}{\mathbb{E}}
\newcommand{\matr}[1]{\mathbf{#1}} 




% color highlighting
\newcommand{\hilight}[1]{\colorbox{yellow}{#1}}


\newcommand{\icol}[1]{% inline column vector
  \left(\begin{smallmatrix}#1\end{smallmatrix}\right)
}

\newcommand{\irow}[1]{% inline row vector
  \begin{smallmatrix} (#1)\end{smallmatrix}
}
\DeclareMathOperator*{\argmin}{\arg\,\min}
\DeclareMathOperator*{\argmax}{\arg\,\max}
